\documentclass[paper=a4]{jlreq}
\usepackage{amsmath}
\usepackage{amssymb}
\usepackage{amsthm}
\usepackage{amsfonts}
\usepackage{mathtools}
\usepackage{graphicx}
\usepackage{multirow}
\usepackage{hyperref}
\usepackage{diffcoeff}
\usepackage{comment}
\usepackage{mhchem}
\usepackage[separate-uncertainty]{siunitx}
\usepackage{newunicodechar}
\usepackage{listings}
\usepackage{float}
\usepackage{lscape}
\usepackage{adjustbox}
\usepackage{tabularx}
\usepackage{booktabs}
\usepackage{longtable}
%%%%%%%%%%%%%%%%%%%%%%%%%%%%%%%%%%%%%%%%%%%%%%%%%
\NewDocumentCommand\degC{}{\ensuremath{^\circ\symup{C}}}
\NewDocumentCommand\abs{m}{\left|#1\right|}
%%%%%%%%%%%%%%%%%%%%%%%%%%%%%%%%%%%%%%%%%%%%%%%%%
\jlreqsetup{
    appendix_counter={
        section={
            value=0,
            the={\Alph{section}}
        },
        table={
            value=0,
            the={\Alph{section}\arabic{table}}
        },
        figure={
            value=0,
            the={\Alph{section}\arabic{figure}}
        }
    },
    appendix_heading={
        section={
            label_format={付録\thesection:}
        }
    }
}

\title{基礎科学実験をスマホで攻略する}
\author{UEC25 \\ xxxxxxx 柴犬被り}
\date{}


\begin{document}

\maketitle\section{Pocket-Ekken}\label{sec:pocket-ekken}\section{実験の目的}\label{sec:実験の目的}\subsection{目的}\label{sec:目的}
このレポジトリは,LaTeXやVSCodeなどの環境構築なしで,スマートフォンを含む環境で簡単に基礎科学実験Aで使用できる実験レポートを制作できるようにすることを目的としている.
\subsection{背景}\label{sec:背景}
基礎科学実験Aで実験レポートを作成する際,VSCodeと$\LaTeX$を使用するのが一般的になっている.
しかし,これらの環境を用意するのは面倒であるし,使い勝手もよいといえない.具体的に例を挙げると
\begin{itemize}
\item 
ローカルでのLaTeXの環境構築には時間がかかり,ある程度のPCスキルが必要である.
\item 
ローカルでの執筆を前提とすると,バックアップに不安がある.
\item 
オンラインエディタを使用する場合,安定したインターネット接続が必要で,サービスが常に利用可能である保証がない[^1]
\item 
基本的に.texを編集するのはコードエディタであり,長文を書くのに向いていない.
\item 
いずれの環境でもスマートフォンでレポートを確認したり編集したりするのは難しい.
\end{itemize}

などの問題がある.
そこで,本レポジトリでは,マークダウンエディタ\href{https://obsidian.md/}{Obsidian}を用いて,スマートフォンを含む様々な環境で簡単に実験レポートを作成できるようにする.
\section{実験の原理}\label{sec:実験の原理}
背景の問題を解決するために,以下の方針を採用した.
\begin{enumerate}
\item 
ObsidianのVaultを配布することで,環境構築を不要にする.
\item 
スマートフォンアプリのObsidianを使用することで,スマートフォンでの編集・確認を可能にする.
\item 
マークダウンファイルをそのままコンパイルできる.texファイルに変換するスクリプトを提供することで,.texファイルの編集を不要にする.
\item 
GitHubと同期することを前提とし,バックアップを容易にするとともに,GitHub Actionsを用いて自動的にPDFを生成する.
\end{enumerate}

構成のイメージは以下の図のようになる.

\begin{figure}[h]
\centering
\includegraphics[width=0.8\textwidth,keepaspectratio]{./images/構成.png}
\caption{構成のイメージ}
\end{figure}

\section{実験方法}\label{sec:実験方法}
ここでは,実際にObsidianを用いて実験レポートを作成する手順を説明する.
まず,スマートフォンまたはPCにObsidianをインストールする.Obsidianは\href{https://obsidian.md/}{公式サイト}からダウンロードできる.
続いて,このレポジトリをダウンロードする.ブラウザで\href{https://github.com/shibadogcap/Pocket-Ekken}{このレポジトリ}にアクセスし,緑の「Code」ボタンをクリックしてリポジトリをダウンロードする.

\begin{figure}[h]
\centering
\includegraphics[width=0.8\textwidth,keepaspectratio]{./images/repo.png}
\caption{GitHubレポジトリ}
\end{figure}


ダウンロードしたzipファイルを解凍し,適当な場所に保存する.
\section{実験結果}\label{sec:実験結果}\section{考察}\label{sec:考察}\section{まとめ}\label{sec:まとめ}\section{参考文献}\label{sec:参考文献}\section{付録}\label{sec:付録}
\end{document}